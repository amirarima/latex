\documentclass{article}
\begin{document}
\maketitle 
\section{Introduction}
An audio signal is a representation of sound, typically as an
electrical voltage. Audio signals have frequencies in the audio
frequency range of roughly 20 to 20,000 Hz (the limits of human
hearing)~\cite{Boney96}.
Audio signals may be synthesized directly, or may originate at a
transducer such as a microphone, musical instrument pickup,
phonograph cartridge, or tape head. Loudspeakers or headphones
convert an electrical audio signal into sound. Digital
representations of audio signals exist in a variety of
formats~\cite{MG,HK,Pan}.
\begin{thebibliography}{100} % 100 is a random guess of the total number of
%references
\bibitem{Boney96} Boney, L., Tewfik, A.H., and Hamdy, K.N., ``Digital
Watermarks for Audio Signals," \emph{Proceedings of the Third IEEE
International Conference on Multimedia}, pp. 473-480, June 1996.
\bibitem{MG} Goossens, M., Mittelbach, F., Samarin, \emph{A LaTeX
Companion}, Addison-Wesley, Reading, MA, 1994.
\bibitem{HK} Kopka, H., Daly P.W., \emph{A Guide to LaTeX},
Addison-Wesley, Reading, MA, 1999.
\bibitem{Pan} Pan, D., ``A Tutorial on MPEG/Audio Compression,"
\emph{IEEE
Multimedia}, Vol.2, pp.60-74, Summer 1998.
\end{thebibliography}
\end{document}
